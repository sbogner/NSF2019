\newpage

%%%%%%%%%%%%%%%%%%%%%%%%%%% start the main stuff here %%%%%%%%%%%%%%%%%%%%

%\setcounter{section}{5}  % set counter so starts at 6 --> F
%\setcounter{page}{1}

\section{Budget Justification}

%
\subsection{Senior Personnel}

This grant proposal covers the research of Scott Bogner (Associate Professor) and Morten Hjorth-Jensen (Professor), both of whom are members of the nuclear theory group at the National Superconducting Cyclotron Laboratory (NSCL) at Michigan State University (MSU). The proposed budget requests two months of summer salary for Bogner, one month of summer salary for Hjorth-Jensen, and full support for two graduate students plus a postdoctoral researcher for the three-year duration of the grant. Both PIs have a demonstrated record of productivity, both in quality and quantity, and are world experts on the methods that form the backbone of this proposal. 

\subsection{Other Personnel} 

We are requesting full support for two graduate (PhD) students at MSU, plus one postdoc who will work on at least one or more of the most computationally demanding projects in our proposal.  
    
There is an excellent pool of graduate students from the Department of Physics and Astronomy at MSU, many of whom come specifically to work in a highly ranked graduate nuclear physics program at the National Superconducting Cyclotron Laboratory. Together, we already have four high-caliber PhD students in our group at MSU.  In the coming academic year, we will also be interacting with many of first and second year graduate students, as Bogner is the Chair of the NSCL graduate student recruitment committee and  Hjorth-Jensen is teaching a popular graduate course on Nuclear Structure.  Therefore, it is likely that we will attract even more students into our group in the coming year.  As detailed in the project description, there is more than enough work in the present proposal to form the basis for two Ph.D. theses in forefront problems in nuclear theory. Indeed, the present proposal could probably support up to three or four PhD theses, in addition to the postdoc we are requesting.

\subsection{Equipment}
The Institute for Cyber Enabled Research (iCER) at MSU is offering a buy-in program for nodes at the HPCC which would give use priority access for their use. The first year of our budget includes \$8.5k to purchase a chassis containing a fat node with Intel Xeon Ivy Bridge processors and 256 GB of RAM. This will also include 1 TB local hard drive storage, with maintenance and basic software provided by the HPCC.

\subsection{Travel}
The PIs and the graduate students are expected to attend the spring APS or fall DNP annual meetings, in addition to relevant domestic conferences such as topical programs at the Institute for Nuclear Theory. Funding is also requested for 1-2  international conferences per year for the PIs. 
Also, as the PIs are co-supervisors of 2 students, funding is also requested for 1-2 trips per year for our shared students to visit Oslo, Norway, where Hjorth-Jensen is based 6 months of the year. 

In general, the theory group at the NSCL recognizes the importance of travel to the
success and professional development of a young physicist's career, and we strongly encourage our graduate students and postdoctoral research associates to attend important workshops and
conferences.  Therefore, \$9,000 in travel funds is requested in the first year of the grant (split 50/50 between domestic and international travel, and adjusted in subsequent years assuming a 3\% inflation rate).

\subsection{Materials and Supplies}
In year 1, we request \$5.5k, primarily to purchase a Dell workstation for the PIs plus sophisticated scientific software packages for use by all personnel under this grant. For instance, a Dell scientific workstation is in the range of \$3k, and specialized software packages such as the Matlab scientific computing program typically costs over \$1200.00 for a single user license, while a Professional Edition Fortran compiler typically prices out in the \$1000 range. We therefore request \$5,500 in funds for Materials and Supplies in year 1. In year 2, we request \$2.0k and adjusted in subsequent years assuming a \%3 inflation rate. This figure will support the replacement of laptop and desktop computers in the group as needed.

\subsection{Other}
This covers graduate student tuition plus fees.

\subsection{Indirect Costs}
Indirect costs are not charged on graduate student tuition and fees. 


