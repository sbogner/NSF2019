\documentclass[11pt]{article}
\usepackage{geometry} % see geometry.pdf on how to lay out the page. There's lots.
\geometry{a4paper} % or letter or a5paper or ... etc
% \geometry{landscape} % rotated page geometry
\usepackage{hyperref}
% See the ``Article customise'' template for come common customisations


%%% BEGIN DOCUMENT
\begin{document}
\centerline{\underline{\Large{Facilities}}} 
\medskip
\medskip
The PIs and other personnel (2 graduate students plus 1 postdoctoral researcher) will benefit from the computational resources at the High Performance Computing Center (HPCC) at Michigan State University. The HPCC provides access to several Linux based clusters with a range of processor and interconnect types. These include:

\begin{enumerate}
\item A 512-core AMD cluster tied together with 1Gb Ethernet and Infiniband. Each of the 128 nodes contains 2 dual-core AMD Opterons running at 2.2 GHz,8 GB of memory, and 100 GB of local disk.
\item A 1024-core Intel cluster tied together with 1Gb Ethernet and Infiniband. Each of the 128  nodes contains 2 quad-core Xeons at 2.3 GHz, 8 GB of memory, and 250 GB of local disk. 

\item A 192-node Intel cluster with 1536 cores. Each node has two four-core Xeon E5620s at 2.4 GHz with 24 GB of memory and 250 GB of local disk.It is connected to the shared Ethernet network and each node has a new quad data rate (QDR) Infiniband connection at 40 gigabits per second.

\item A fat node cluster consisting of four Sun Fire X4600 M2 nodes with 256 GB RAM and 32 AMD Opteron cores and one node with 16 cores and 128 GB of RAM. 

\item  A 32-node GPU cluster. Each node has 18 GB of RAM, 2 Intel E5530 processors, 250 GB of local disk, and 2 Nvidia Tesla M1060 gpu cards. 

\end{enumerate}

The PIs and other personnel will benefit from good working conditions at Michigan State University, including offices and administrative support in the Theory wing of the NSCL.
\end{document}
