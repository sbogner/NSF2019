
\documentclass[12pt]{article}
\usepackage{geometry} % see geometry.pdf on how to lay out the page. There's lots.
\geometry{a4paper} % or letter or a5paper or ... etc
% \geometry{landscape} % rotated page geometry

% See the ``Article customise'' template for come common customisations

\title{Budget Impact}
\date{} % delete this line to display the current date

%%% BEGIN DOCUMENT
\begin{document}

\maketitle
Due to the 38.4\% reduction in the revised budget, the current award will support a postdoctoral researcher for 12 months, as opposed to the 36 months requested in our proposal.  This will have especially strong negative consequences for the projects outlined in the section {\bf Optimized forces and estimation of theoretical uncertainties}, as this was the area for which we anticipated having the services of a postdoc for the duration of the grant.  

This will also indirectly impact the research outlined in the other sections of our proposal, which we envisioned as providing a number of possible Ph.D. projects, as we will be forced to shift graduate students to work on the optimization/uncertainty quantification projects.  As a consequence of the reduction in funding, the PIs estimate that the following research projects will be eliminated or severely reduced in scope:
\begin{itemize}
\item {\bf Microscopically-based energy density functionals for nuclei.}  Our project to use first-principle IM-SRG and CC ground state calculations of medium-mass nuclei to construct microscopic energy density functionals via the method of adiabatic-connection curves will be eliminated entirely. This project was originally intended as a possible Ph.D. project. The elimination is necessary, as we will be diverting a graduate student(s) to work on the projects discussed below, which were originally designed to be driven by a full-time postdoc.

\item{\bf Nuclear symmetry energy and transport properties.} For the same reasons as the previous bullet, our plan to use CC/IM-SRG/FCIQMC calculations of infinite matter to extract i) information about the symmetry energy and ii) determine Landau parameters relevant to transport theory will be substantially delayed or eliminated.  This adversely impacts possible future collaborations with NSCL colleagues Profs. Danielewicz and Nunes to study the symmetry energy and transport properties of nucleonic matter.

\item{\bf Optimization of chiral EFT potentials.} The proposed optimization of chiral EFT forces using advanced statistical packages by extending the fit data to include light- and medium-mass nuclei will be severely reduced in scope, as it was originally intended to be one of the primary projects for a postdoctoral researcher. The PI's will continue to work part-time on this problem in collaboration with local students and colleagues at ORNL, but progress will be slowed by the fact that there will not be a half- or full postdoc devoted to the project.  
\item{\bf Optimization of shell model interactions.} For the same reasons as the previous bullet point, our plans to use microscopically-derived shell model interactions via CC/IM-SRG theory in conjunction with advanced statistical optimization packages to develop improved semi-phenomenological shell model Hamiltonians will be severely reduced in scope. 

\item{\bf Development of a scientific library.} Our proposed plan to build a widely accessible software library that integrates various many-body methods (CC, IM-SRG, and FCIQMC) with state-of-the-art optimization software (e.g., POUNDERS) will be eliminated, as it was originally targeted to occur near the end of the award period, assuming the services of a full-time postdoc for the duration of the award.

\end{itemize}

Additionally, the revised budget eliminates the \$8,500 requested for an HPC buy-in on a new fat node machine, which would give our research group preferential access. This will globally slow advances in our IM-SRG related projects, as such calculations are most naturally carried out on large memory machines.  
\end{document}