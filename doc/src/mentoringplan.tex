
\documentclass[11pt]{article}
\usepackage{geometry} % see geometry.pdf on how to lay out the page. There's lots.
\geometry{a4paper} % or letter or a5paper or ... etc
% \geometry{landscape} % rotated page geometry
\usepackage{hyperref}
% See the ``Article customise'' template for come common customisations


%%% BEGIN DOCUMENT
\begin{document}
\centerline{\underline{\Large{Postdoc Mentoring Plan}}} 
\medskip
\medskip
The mentoring of the postdoc located at the NSCL will be fully incorporated in the existing mentoring program of the NSCL. At any time there are about 15 experimental and theoretical postdocs at the NSCL. It is an ideal research environment which will enable the postdoc to work and interact with students, postdocs and more senior researches with diverse backgrounds on a daily basis. A few years ago the NSCL strengthened and improved the mentoring of graduate students and postdocs at the NSCL with the appointment of an Associate Director for Education. A formal half-year review of all postdocs was initiated. It begins with an entry interview where the postdoc and supervisor discuss expectations and future career plans of the postdoc guided by a form containing standard topics and questions related to the research activities. This initial discussion is then followed but by similar meetings every 6 months. These reviews occur in April and October of each year and are required and monitored by the Associate Director for Education.

The Associate Director for Education meets with all NSCL postdocs on a monthly basis to keep them informed about NSCL related news and activities. During these meetings the representative of the various committees also report back to the group. Postdocs also maintain their own website, where they share experiences, hints and collect and list job openings. Postdocs are also encouraged to participate in activities and programs offered by the University and the professional societies. For example, recent postdocs have been participating in an MSU sponsored Work/Life balance workshop and the APS professional development workshops. The NSCL especially focuses on career planning. Every semester two seminar speakers are especially chosen to present a broad view of career opportunities outside the traditional academic track. Yearly alumni events offer additional interaction of current students and postdocs with successful  NSCL alumni in a variety of careers.  The NSCL alumni contact list currently contains the names of 250 alumni who offered to be contacted by students and postdocs for career advice (\url{http://www.nscl.msu.edu/ourlab/alumni}).  The list can be filtered by profession and geographic distribution.

Finally, the postdocs are an integral part of all laboratory social activities at the NSCL which include the Tuesdays coffee and bagel, Thursdays ice cream social, as well as summer BBQs and welcome receptions. These activities further foster the interaction between all laboratory employees.




\end{document}