\documentclass[10pt]{article}
\usepackage{geometry} % see geometry.pdf on how to lay out the page. There's lots.
\geometry{a4paper} % or letter or a5paper or ... etc
% \geometry{landscape} % rotated page geometry

% See the ``Article customise'' template for come common customisations

%%% BEGIN DOCUMENT
\begin{document}
\title{Computational Nuclear Many-Body Physics}
\date{}
\maketitle

Nuclear physics plays a key role in our quest to understand  the
Universe, addressing fundamental scientific questions like: 1) How did matter come
  into being and how does it evolve? 2) How does subatomic
  matter organize itself and what phenomena emerge? 3) Are
  the fundamental interactions that are basic to the structure of
  matter fully understood?, and 4) How can the knowledge and
  technological progress provided by nuclear physics best be used to
  benefit society?  In recent years, researchers
have made remarkable progress in our fundamental understanding of the
complex and fascinating system that is the
nucleus. This progress has been driven
by new theoretical insights and increased computational power, as well
as by experimental access to new isotopes with a large excess of
neutrons or protons.  The latter involves large societal investments in 
scientific forefront experimental facilities like the Facility of Radioactive Ion Beams which is being built at Michigan State University in the U.S.

However, while much has been learned so far
about nuclear systems and associated phenomena, much remains to be
understood.
This application aims thus at further advancing our basic understanding of subatomic matter
by addressing many of the above fundamental questions. For
nuclear theorists, the challenge is to develop a comprehensive and
unified description of nuclei and their reactions, grounded in the
fundamental interactions between the constituent nucleons with
quantifiable uncertainties to maximize predictive power. To model such systems according to the 
laws of motion and the underlying nuclear forces, requires the development of sophisticated 
physical and mathematical algorithms.
To address
this challenge, we propose to develop a toolbox  of methods for 
dealing with many strongly interacting particles 
(protons and neutrons) 
capable of treating a wide variety of nuclear systems, ranging from
stable closed-shell nuclei and nuclear matter as seen in for example neutron stars to exotic
loosely-bound neutron and proton rich nuclei far from shell
closures. The toolbox will make use of the best possible microscopic
inputs and
will be built around thoroughly modern many-particle methods such as
coupled cluster, in-medium similarity renormalization group, and full
configuration interaction quantum stochastic (Monte Carlo) methods.  A premium will
be placed on developing reliable theoretical error bars, which stem in
part from truncation errors used to derive nuclear
forces, uncertainties in the fitted parameters of the input
interactions, truncated renormalization group (RG) evolution to
``soften'' the input Hamiltonian, basis-set truncation errors, and
truncation errors in the particular level of many-body approximation.
By developing powerful many-body methods capable of treating a wide
range of nuclear systems with controlled uncertainties, we will be one
step closer to being able to answer the above fundamental questions of nuclear
physics.



\end{document}




