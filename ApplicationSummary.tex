\section{Project Summary}
\label{ProjectSummary}

\noindent
{\bf Title of Project: Ab initio Nuclear Theory: From Nuclei to Neutron Stars}  

\noindent
{\bf Principal Investigators:}  
S.~K.\ Bogner and M.~Hjorth-Jensen

\medskip
Michigan State University (MSU) and the National Superconducting
Cyclotron Laboratory (NSCL) are submitting a renewal proposal for a
three-year grant supporting research conducted by two faculty members
in nuclear theory, Associate Professor Scott Bogner and Professor
Morten Hjorth-Jensen. In cooperation with national and international
collaborators, we propose to develop a nuclear many-body modeling
infrastructure based on state-of-the-art \emph{ab initio}
Coupled-Cluster (CC) Theory and In-Medium Similarity Renormalization
Group (IMSRG) methods. The goal is to develop a comprehensive \emph{ab
  initio} framework that is capable of providing controlled and
predictive calculations for nucleonic matter over a wide range of
conditions, from nuclei to dense matter and neutron stars. The
proposed activities will be closely linked with ongoing experimental
programs in low-energy nuclear physics nationally and worldwide as
well.


{\bf Intellectual merit:} For nuclear theorists, the overarching
challenge is to develop a comprehensive description of nuclei and
their reactions, grounded in the fundamental interactions between the
constituent nucleons with quantifiable uncertainties to maximize
predictive power.  As experimental frontiers have shifted to the study
of rare isotopes, the predictive power of successful phenomenological
approaches like the shell model and density functional theory is
challenged by the scarcity of nearby experimental data to constrain
model parameters. Therefore, it is expected that \emph{ab initio}
methods will play an increasingly prominent role to help improve the
predictive power of such ``data driven'' methods as experiment moves
deeper into largely unexplored regions of the nuclear chart.

To address this challenge, we propose to develop and apply
complementary many-body methods to a wide variety of nuclear systems,
ranging from stable closed-shell nuclei and homogenous dense nuclear
matter to exotic loosely-bound neutron and proton rich nuclei far from
shell closures. The proposed research will be built around thoroughly
modern \emph{ab initio} many-body methods such as coupled cluster
theory and the in-medium similarity renormalization group.  A premium
will be placed on developing reliable theoretical error bars, which
stem in part from truncation errors of the input nuclear forces,
uncertainties in the fitted parameters of the input interactions,
basis-set truncation errors, and truncation errors in the particular
level of many-body approximation.  By developing powerful many-body
methods capable of treating a wide range of nuclear systems with
controlled uncertainties, we will be one step closer to being able to
answer fundamental questions of nuclear physics, such as What are
limits of nuclear stability?, What are the masses of neutrinos?, What
is the equation of state for nuclear matter?, and How does nuclear
structure emerge from underlying two- and three-nucleon (and higher)
interactions?

{\bf Broader impacts:} The training received by undergraduates,
graduate students, and postdoctoral research associates in carrying
out the proposed activities contributes directly to the building of a
diverse scientific workforce. The present proposal has a large
educational component, being part of the nuclear physics program at
NSCL/MSU that was recently ranked number one in the country. The mix
of analytical and numerical computations our students must employ is
an excellent preparation for both academic and industrial research. The
PIs are committed to diversity in science.





