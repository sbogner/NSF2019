
%test commit

\section{Project Summary}
\label{ProjectSummary}

\noindent
{\bf Title of Project:}  Computational Nuclear Many-Body Physics

\noindent
{\bf Principal Investigators:}  
S.~K.\ Bogner and M.~Hjorth-Jensen

\medskip
Michigan State University (MSU) and the National Superconducting
Cyclotron Laboratory (NSCL) are submitting a renewal proposal for a
three-year grant supporting research conducted by two faculty members
in nuclear theory, Associate Professor Scott Bogner and Professor
Morten Hjorth-Jensen. In cooperation with national and international
collaborators, we propose to develop a nuclear many-body modeling
infrastructure consisting of several state-of-the-art many-body
methods, implementations, and applications. Through this
infrastructure new computational methodologies will be introduced, and
by integrating the methods into a software library, the acquired
knowledge will be accumulated and made easily accessible for both
theorists and experimentalists.  The proposed activities will be
closely linked with ongoing experimental programs in low-energy
nuclear physics nationally and worldwide as well.





% Combined with a broad national and international cooperation, we propose to develop a nuclear many-body modeling infrastructure consisting of several state-of-the-art many-body methods, implementations, and applications. Through this infrastructure new computational
%methodologies will be introduced, and by integrating the methods into
%a software library, the acquired knowledge will be accumulated and
%made easily accessible for both theorists and experimentalists.  The proposed activities will be closely linked with ongoing experimental programs in low-energy nuclear physics nationally and worldwide as well. 


%To achieve this
%goal, a vibrant interplay between theory and experiment is
%needed. This interplay should match the research conducted at present
%and planned experimental facilities, where a primary aim is to study
%short-lived isotopes that convey crucial information about
%the stability of nuclear matter and the origin of elements, but are
%difficult to study experimentally due to their extremely short
%ifetimes and small production cross-sections. 



{\bf Intellectual merit:} To understand why nucleonic matter is
stable, how it comes into being, how it evolves and organizes itself
and what phenomena emerge, is one of the overarching aims and
intellectual challenges of basic research in nuclear physics. For
nuclear theorists, the challenge is to develop a comprehensive and
unified description of nuclei and their reactions, grounded in the
fundamental interactions between the constituent nucleons with
quantifiable uncertainties to maximize predictive power. To address
this challenge, we propose to develop a nuclear many-body toolbox
capable of treating a wide variety of nuclear systems, ranging from
stable closed-shell nuclei and homogenous nuclear matter to exotic
loosely-bound neutron and proton rich nuclei far from shell
closures. The toolbox will make use of the best possible microscopic
inputs (e.g., chiral effective field theory (EFT) two-nucleon and three-nucleon
interactions optimized to data using advanced statistical tools), and
will be built around thoroughly modern many-body methods such as
coupled cluster, in-medium similarity renormalization group, and full
configuration interaction quantum monte carlo methods.  A premium will
be placed on developing reliable theoretical error bars, which stem in
part from truncation errors of the input EFT used to derive nuclear
forces, uncertainties in the fitted parameters of the input
interactions, truncated renormalization group (RG) evolution to
``soften'' the input Hamiltonian, basis-set truncation errors, and
truncation errors in the particular level of many-body approximation.
By developing powerful many-body methods capable of treating a wide
range of nuclear systems with controlled uncertainties, we will be one
step closer to being able to answer fundamental questions of nuclear
physics, such as What are limits of nuclear stability?, What are the
masses of neutrinos?, What is the equation of state for nuclear
matter?, and How does nuclear structure emerge from underlying two-
and three-nucleon (and higher) interactions?

{\bf Broader impacts:} The training received by undergraduates,
graduate students, and postdoctoral research associates in carrying
out the proposed activities contributes directly to the building of a
diverse scientific workforce. The present proposal has a large
educational component, being part of the nuclear physics program at
NSCL/MSU that was recently ranked number one in the country. The mix
of analytical and numerical computations our students must employ is
an excellent preparation for both academic and industrial research. The
PIs are committed to diversity in science.


